\documentclass[letterpaper,10pt,titlepage]{article}
%
%\pagestyle{headings}
%
\usepackage{amsmath}
\usepackage{amsfonts}
\usepackage{amssymb}
\usepackage[ansinew]{inputenc}
\usepackage[OT1]{fontenc}
\usepackage{graphicx}
%
\begin{document}
%-----------------------------------------------------------------------------------
\title{Notes from Renewing My Private Pilot's License in 2020}
\author{\vspace{4cm}\\David T. Ashley\\\texttt{dashley@gmail.com}\\\vspace{4cm}}
\date{\small{\LaTeX{} Compilation Date: \today{}}}
\maketitle

%%%%%%%%%%%%%%%%%%%%%%%%%%%%%%%%%%%%%%%%%%%%%%%%%%%%%%%%%%%%%%%%%%%%%%%%%%%%%%%
%
\begin{abstract}
This document contains my notes from renewing my PPL (private pilot's license)
in 2020.
\end{abstract}

\clearpage{}
\pagenumbering{roman}    %No page number on table of contents.
\tableofcontents{}
\clearpage{}
\listoffigures
\clearpage{}

%%%%%%%%%%%%%%%%%%%%%%%%%%%%%%%%%%%%%%%%%%%%%%%%%%%%%%%%%%%%%%%%%%%%%%%%%%%%%%%
%Force the page number to 1.  We don't want to number the table of contents
%page.
%
\setcounter{page}{1}
\pagenumbering{arabic}
%%%%%%%%%%%%%%%%%%%%%%%%%%%%%%%%%%%%%%%%%%%%%%%%%%%%%%%%%%%%%%%%%%%%%%%%%%%%%%%

\section{Introduction and Overview}
\label{siov0}

I earned a PPL in 2005.  In 2020 (after 15 years of not flying), I decided
to renew the PPL.  This document contains my notes from the process.

I do not expect that these notes will be useful to anyone except me.  Because I
develop software for a living, I'm very familiar with version control systems,
and this document is kept under \emph{git} at \emph{GitHub}, where it will
be indexed by search
engines.  If anyone finds this document useful in any way, please write me
at \emph{dashley@gmail.com}---I might add a \emph{Comments and Feedback}
section in this document, or expand some sections with feedback and discussion.

The primary purpose of keeping this document under version control is
redundancy and the ability to work seamlessly from multiple computers.  The
primary purpose does not involve traditional version control features that are
useful for software development.

This document is typeset using an archaic word processing system, \LaTeX{}.
I will try to keep a \emph{.pdf} version in \emph{git} with the text input
files.

I received input from 3 flight instructors as I sought to renew my PPL.
To protect their privacy, their names are not included in the document.
These flight instructors were:

\begin{itemize}
   \item CFI\textsubscript{1}, the flight instructor who trained me in 2020
         at Solo Aviation in Ann Arbor, Michigan.
   \item CFI\textsubscript{2}, the primary flight instructor who trained me
         in 2004/2005.
   \item CFI\textsubscript{3}, a secondary flight instructor who trained me
         in 2004/2005.
\end{itemize}


%%%%%%%%%%%%%%%%%%%%%%%%%%%%%%%%%%%%%%%%%%%%%%%%%%%%%%%%%%%%%%%%%%%%%%%%%%%%%%%

\section{Notes from Reviewing \emph{SAFO Memo Dated 8/30/16}}
\label{ssaf0}

PPL standards were changed to perform slow flight above the threshold
of activating the stall warning.

%%%%%%%%%%%%%%%%%%%%%%%%%%%%%%%%%%%%%%%%%%%%%%%%%%%%%%%%%%%%%%%%%%%%%%%%%%%%%%%

\section{Notes from Reviewing \emph{2016/2020 Airplane Flying Handbook}}
\label{snra0}

I used a mixture of the 2016 and the 2020 versions of the \emph{Airplane Flying
Handbook}.  The 2016 version was in \emph{.pdf} format, downloaded from the
FAA's website.  The 2020 version was ordered from \emph{Amazon} for my
\emph{Kindle}.  I do not expect there will be any meaningful differences between
the two versions, and if there are any recent rule changes so that there is
different information between the two versions, I expect I will become aware
of this as I read the \emph{Gleim} and other materials.

%%%%%%%%%%%%%%%%%%%%%%%%%%%%%%%%%%%%%%%%%%%%%%%%%%%%%%%%%%%%%%%%%%%%%%%%%%%%%%%

\subsection{Notes from Reviewing \emph{Chapter 1:  Introduction to Flight
            Training}}
\label{snra0:sift0}

\begin{itemize}
\item CFR was formerly referred to as the FAR.
\item Title 14 of CFR (``14 CFR'') is \emph{Aeronautics and Space}.
\item 14 CFR Part 91 would be a typical citation.
\item AFM/POH, airplane flight manual / pilot's operating handbook lists items
      required for airworthiness.
\item Flight Standards Service (AFS).  FSDO is the public interface.
\item NAS:  National Airspace System.
\item Collision Avoidance
      \begin{itemize}
	  \item ``See and avoid''.
	  \item Collisions most typically occur:
	        \begin{itemize}
			\item Within 5 miles of an airport.
			\item Near navigation aids.
			\item With good visibility.
			\end{itemize}
	  \end{itemize}
	  \item Stay alert to all traffic in field of vision.
	  \item Scan periodically, all the time.
	  \item Scan technique:
	        \begin{itemize}
			\item Shift glances.
			\item Refocus at intervals.
			\item Short, regularly spaced eye movements, $\leq$10 degrees, and observe
			      for at least 1 second.
			\end{itemize}
	  \item Clearing procedure:
	        \begin{itemize}
			\item Question for CFI\textsubscript{1}:  what is the best procedure?
			\end{itemize}
	  \item Stall awareness:
	        \begin{itemize}
			\item Critical AOA is typically 16 to 20 degrees.
            \item Low speed is not necessary to produce a stall.
			\end{itemize}
	  \item Use of checklists.
      \item Positive transfer of controls:	  
	        \begin{itemize}
			\item Why was CFI\textsubscript{1} unhappy with ``your airplane''?
			\end{itemize}
\end{itemize}

%%%%%%%%%%%%%%%%%%%%%%%%%%%%%%%%%%%%%%%%%%%%%%%%%%%%%%%%%%%%%%%%%%%%%%%%%%%%%%%

\subsection{Notes from Reviewing \emph{Chapter 2:  Ground Operations}
            (Reviewed 7/23/2020)}
\label{snra0:sgop0}

\begin{itemize}
\item Required documents:
	  \begin{itemize}
	  \item Airworthiness certificate.
	  \item Registration.
	  \item Radio station license (flights outside US, weight $>$12,500 lbs.).
	  \item Operating limitations / POH.
	  \item Official weight and balance.
	  \item Compass deviation card.
	  \item External data plate.
	  \end{itemize}
\item Fuel vents are important---will result in fuel starvation if not operational.
\item Rush of air when opening fuel cap is likely sign of fuel vent problem.
\item SRM:  single-pilot resource management.
\item Review airport diagram in advance, for safety and lack of unnecessary distractions.
\item Anti-collision lights must go on before engine start.
\item Look around while taxiing.  Look from side to side.
\item Wind on the ground, turn into, dive away from.  (Question:  for turn into,
      how to make elevator neutral and how important?)
\item Before takeoff check:
	  \begin{itemize}
	  \item Straighten nosewheel.
	  \item Run-up into wind to minimize the possibility of overheating the
	        cylinders/engine.
	  \end{itemize}
\end{itemize}

%%%%%%%%%%%%%%%%%%%%%%%%%%%%%%%%%%%%%%%%%%%%%%%%%%%%%%%%%%%%%%%%%%%%%%%%%%%%%%%

\subsection{Notes from Reviewing \emph{Chapter 3:  Basic Flight Maneuvers}
            (Reviewed 7/25/2020)}
\label{snra0:sgop0}

\begin{itemize}
\item Light touch on all controls, including rudder pedals.  Must feel the
      resistance.
\item Straight and level flight:
	  \begin{itemize}
	  \item Need to try looking at wingtips to judge equal distance to horizon.  I do
	        not do this.  (According to AFH, need to do this to level the plane.)
	  \end{itemize}
\item Trim control.
	  \begin{itemize}
	  \item Don't fly the airplane with the trim.
	  \item Make changes with controls, then adjust trim to relieve pressures.
	  \end{itemize}

\item Level turns.
	  \begin{itemize}
	  \item In a constant-altitude constant-airspeed turn, must add power and pull yoke.
	  \item Must use rudder to stay coordinated.  Wings generate yaw forces in a turn.
	  \item Uncoordinated flight encourages spins.
	  \item ``slip'' versus ``skid''.
	  \item At a given angle of bank, higher airspeed = bigger radius of turn.
	  \item Don't lean head/body during turns.
	  \item May want to re-trim during steep turns.
	  \item A rule of thumb is to lead the rollout by half the bank angle.  In a 30-degrees
	        bank turn, begin to roll out 15 degrees early.
	  \item May need to work on rudder motion/synchronization as enter and exit turns.
	  \item Must remember clearing turns.
	  \end{itemize}
\item Climbs and climbing turns.
	  \begin{itemize}
	  \item Climb is limited by excess thrust.
	  \item A normal climb is cruise climb, $v > V_Y$, for better engine cooling.
	  \item Definition of $V_Y$, $V_X$.
	  \item With higher altitude, $V_Y$ decreases and $V_X$ increases.  (Why?)
	  \item $V_X = V_Y$ occurs at absolute ceiling of airplane.  (Why?)
	  \item Right rudder, P-factor.
	  \item Level off should begin around 10 percent of the rate of climb.  500 ft/min climb
	        means begin level off 50 feet early.
	  \item Keep climb power briefly after leveling off so as to reach cruise speed.
	  \end{itemize}
\item Climbing turns.
	  \begin{itemize}
	  \item Cannot maintain same pitch/airspeed as level climb.
	  \item Steep banks take away from climb.
	  \item In a climbing turn, bank should be shallow.
	  \item Should maintain coordinated flight, constant airspeed, constant rate of turn.
	  \end{itemize}
\item Descending turns.
	  \begin{itemize}
	  \item Partial power descent:  typically 500 ft/min.
	  \item Minimum safe airspeed descent:  typically obstacle clearance on landing.
	  \item Emergency descent:  procedure depends on the model of airplane (gear, flap settings, other elements of procedure).
	  \end{itemize}
\item Glides
	  \begin{itemize}
	  \item Use 10 percent of rate of descent as a guide for when to begin leveling.
	  \item Best glide speed:  max L/D.
	  \item Left rudder likely necessary.
	  \item Flight control deflection likely greater, due to reduced airflow over surfaces.
	  \item Minimum sink speed:  a few knots less than the best glide speed, typically.  Seldom published.
	        Appropriate for certain maneuvers like ditching into water.
	  \item Never attempt to stretch a glide.
	  \end{itemize}
\item Gliding turns.
	  \begin{itemize}
	  \item More back pressure on elevator required than during powered flight.
	  \item Less rudder pressure required because of less airflow.  Deflection may be similar, but less
	        pressure required to achieve the deflection.
	  \item Hold nose up until best glide speed is reached, and only then allow further descent.
	  \item Must lower nose slightly at gliding turn entry (must have lesser pitch than glide).
	  \end{itemize}
\end{itemize}


%%%%%%%%%%%%%%%%%%%%%%%%%%%%%%%%%%%%%%%%%%%%%%%%%%%%%%%%%%%%%%%%%%%%%%%%%%%%%%%

\section{Notes from Cessna 172R POH}
\label{snrc0}

TBD.


%%%%%%%%%%%%%%%%%%%%%%%%%%%%%%%%%%%%%%%%%%%%%%%%%%%%%%%%%%%%%%%%%%%%%%%%%%%%%%%

\section{Notes about Obtaining Third-Class Medical Certificate}
\label{stcm0}

TBD.


%%%%%%%%%%%%%%%%%%%%%%%%%%%%%%%%%%%%%%%%%%%%%%%%%%%%%%%%%%%%%%%%%%%%%%%%%%%%%%%
\end{document}
