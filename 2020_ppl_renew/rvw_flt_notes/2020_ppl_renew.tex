\documentclass[letterpaper,10pt,titlepage]{article}
%
%\pagestyle{headings}
%
\usepackage{amsmath}
\usepackage{amsfonts}
\usepackage{amssymb}
\usepackage[ansinew]{inputenc}
\usepackage[OT1]{fontenc}
\usepackage{graphicx}
\usepackage{makeidx}
\makeindex{}
%
\begin{document}
%-----------------------------------------------------------------------------------
\title{Notes from Renewing My Private Pilot's License in 2020}
\author{\vspace{4cm}\\David T. Ashley\\\texttt{dashley@gmail.com}\\\vspace{4cm}}
\date{\small{\LaTeX{} Compilation Date: \today{}}}
\maketitle

%%%%%%%%%%%%%%%%%%%%%%%%%%%%%%%%%%%%%%%%%%%%%%%%%%%%%%%%%%%%%%%%%%%%%%%%%%%%%%%
%
\begin{abstract}
This document contains my notes from renewing my PPL (private pilot's license)
in 2020.
\end{abstract}

\clearpage{}
\pagenumbering{roman}    %No page number on table of contents.
\tableofcontents{}
\clearpage{}
%\listoffigures
%\clearpage{}

%%%%%%%%%%%%%%%%%%%%%%%%%%%%%%%%%%%%%%%%%%%%%%%%%%%%%%%%%%%%%%%%%%%%%%%%%%%%%%%
%Force the page number to 1.  We don't want to number the table of contents
%page.
%
\setcounter{page}{1}
\pagenumbering{arabic}
%%%%%%%%%%%%%%%%%%%%%%%%%%%%%%%%%%%%%%%%%%%%%%%%%%%%%%%%%%%%%%%%%%%%%%%%%%%%%%%

\section{Introduction and Overview}
\label{siov0}

I earned a PPL in 2005.  In 2020 (after 15 years of not flying), I decided
to renew the PPL.  This document contains my notes from the process.

I do not expect that these notes will be useful to anyone except me.  Because I
develop software for a living, I'm very familiar with version control systems,
and this document is kept under \emph{git} at \emph{GitHub}, where it will
be indexed by search
engines.  If anyone finds this document useful in any way, please write me
at \emph{dashley@gmail.com}---I might add a \emph{Comments and Feedback}
section in this document, or expand some sections with feedback and discussion.

The primary purpose of keeping this document under version control is
redundancy and the ability to work seamlessly from multiple computers.  The
primary purpose does not involve traditional version control features that are
useful for software development.

This document is typeset using an archaic word processing system, \LaTeX{}.
I will try to keep a \emph{.pdf} version in \emph{git} with the text input
files.

I received input from 3 flight instructors as I sought to renew my PPL.
To protect their privacy, their names are not included in the document.
These flight instructors were:

\begin{itemize}
   \item CFI\textsubscript{1}, the flight instructor who trained me in 2020
         at Solo Aviation in Ann Arbor, Michigan.
   \item CFI\textsubscript{2}, the primary flight instructor who trained me
         in 2004/2005.
   \item CFI\textsubscript{3}, a secondary flight instructor who trained me
         in 2004/2005.
\end{itemize}


%%%%%%%%%%%%%%%%%%%%%%%%%%%%%%%%%%%%%%%%%%%%%%%%%%%%%%%%%%%%%%%%%%%%%%%%%%%%%%%

\section{Notes from Reviewing \emph{SAFO Memo Dated 8/30/16}}
\label{ssaf0}

PPL standards were changed to perform slow flight above the threshold
of activating the stall warning.

%%%%%%%%%%%%%%%%%%%%%%%%%%%%%%%%%%%%%%%%%%%%%%%%%%%%%%%%%%%%%%%%%%%%%%%%%%%%%%%

\section{Notes from Reviewing \emph{2016/2020 Airplane Flying Handbook}}
\label{snra0}

I used a mixture of the 2016 and the 2020 versions of the \emph{Airplane Flying
Handbook}.  The 2016 version was in \emph{.pdf} format, downloaded from the
FAA's website.  The 2020 version was ordered from \emph{Amazon} for my
\emph{Kindle}.  I do not expect there will be any meaningful differences between
the two versions, and if there are any recent rule changes so that there is
different information between the two versions, I expect I will become aware
of this as I read the \emph{Gleim} and other materials.

%%%%%%%%%%%%%%%%%%%%%%%%%%%%%%%%%%%%%%%%%%%%%%%%%%%%%%%%%%%%%%%%%%%%%%%%%%%%%%%

\subsection{Notes from Reviewing \emph{Chapter 1:  Introduction to Flight
            Training}}
\label{snra0:sift0}

\begin{itemize}
\item CFR was formerly referred to as the FAR.
\item Title 14 of CFR (``14 CFR'') is \emph{Aeronautics and Space}.
\item 14 CFR Part 91 would be a typical citation.
\item AFM/POH, airplane flight manual / pilot's operating handbook lists items
      required for airworthiness.
\item Flight Standards Service (AFS).  FSDO is the public interface.
\item NAS:  National Airspace System.
\item Collision Avoidance
      \begin{itemize}
	  \item ``See and avoid''.
	  \item Collisions most typically occur:
	        \begin{itemize}
			\item Within 5 miles of an airport.
			\item Near navigation aids.
			\item With good visibility.
			\end{itemize}
	  \end{itemize}
	  \item Stay alert to all traffic in field of vision.
	  \item Scan periodically, all the time.
	  \item Scan technique:
	        \begin{itemize}
			\item Shift glances.
			\item Refocus at intervals.
			\item Short, regularly spaced eye movements, $\leq$10 degrees, and observe
			      for at least 1 second.
			\end{itemize}
	  \item Clearing procedure:
	        \begin{itemize}
			\item Question for CFI\textsubscript{1}:  what is the best procedure?
			\end{itemize}
	  \item Stall awareness:
	        \begin{itemize}
			\item Critical AOA is typically 16 to 20 degrees.
            \item Low speed is not necessary to produce a stall.
			\end{itemize}
	  \item Use of checklists.
      \item Positive transfer of controls:	  
	        \begin{itemize}
			\item Why was CFI\textsubscript{1} unhappy with ``your airplane''?
			\end{itemize}
\end{itemize}

%%%%%%%%%%%%%%%%%%%%%%%%%%%%%%%%%%%%%%%%%%%%%%%%%%%%%%%%%%%%%%%%%%%%%%%%%%%%%%%

\subsection{Notes from Reviewing \emph{Chapter 2:  Ground Operations}
            (Reviewed 7/23/2020)}
\label{snra0:sgop0}

\begin{itemize}
\item Required documents:
	  \begin{itemize}
	  \item Airworthiness certificate.
	  \item Registration.
	  \item Radio station license (flights outside US, weight $>$12,500 lbs.).
	  \item Operating limitations / POH.
	  \item Official weight and balance.
	  \item Compass deviation card.
	  \item External data plate.
	  \end{itemize}
\item Fuel vents are important---will result in fuel starvation if not operational.
\item Rush of air when opening fuel cap is likely sign of fuel vent problem.
\item SRM:  single-pilot resource management.
\item Review airport diagram in advance, for safety and lack of unnecessary distractions.
\item Anti-collision lights must go on before engine start.
\item Look around while taxiing.  Look from side to side.
\item Wind on the ground, turn into, dive away from.  (Question:  for turn into,
      how to make elevator neutral and how important?)
\item Before takeoff check:
	  \begin{itemize}
	  \item Straighten nosewheel.
	  \item Run-up into wind to minimize the possibility of overheating the
	        cylinders/engine.
	  \end{itemize}
\end{itemize}

%%%%%%%%%%%%%%%%%%%%%%%%%%%%%%%%%%%%%%%%%%%%%%%%%%%%%%%%%%%%%%%%%%%%%%%%%%%%%%%

\subsection{Notes from Reviewing \emph{Chapter 3:  Basic Flight Maneuvers}
            (Reviewed 7/25/2020)}
\label{snra0:sbfm0}

\begin{itemize}
\item Light touch on all controls, including rudder pedals.  Must feel the
      resistance.
\item Straight and level flight:
	  \begin{itemize}
	  \item Need to try looking at wingtips to judge equal distance to horizon.  I do
	        not do this.  (According to AFH, need to do this to level the plane.)
	  \end{itemize}
\item Trim control.
	  \begin{itemize}
	  \item Don't fly the airplane with the trim.
	  \item Make changes with controls, then adjust trim to relieve pressures.
	  \end{itemize}

\item Level turns.
	  \begin{itemize}
	  \item In a constant-altitude constant-airspeed turn, must add power and pull yoke.
	  \item Must use rudder to stay coordinated.  Wings generate yaw forces in a turn.
	  \item Uncoordinated flight encourages spins.
	  \item ``slip'' versus ``skid''.
	  \item At a given angle of bank, higher airspeed = bigger radius of turn.
	  \item Don't lean head/body during turns.
	  \item May want to re-trim during steep turns.
	  \item A rule of thumb is to lead the rollout by half the bank angle.  In a 30-degrees
	        bank turn, begin to roll out 15 degrees early.
	  \item May need to work on rudder motion/synchronization as enter and exit turns.
	  \item Must remember clearing turns.
	  \end{itemize}
\item Climbs and climbing turns.
	  \begin{itemize}
	  \item Climb is limited by excess thrust.
	  \item A normal climb is cruise climb, $v > V_Y$, for better engine cooling.
	  \item Definition of $V_Y$, $V_X$.
	  \item With higher altitude, $V_Y$ decreases and $V_X$ increases.  (Why?)
	  \item $V_X = V_Y$ occurs at absolute ceiling of airplane.  (Why?)
	  \item Right rudder, P-factor.
	  \item Level off should begin around 10 percent of the rate of climb.  500 ft/min climb
	        means begin level off 50 feet early.
	  \item Keep climb power briefly after leveling off so as to reach cruise speed.
	  \end{itemize}
\item Climbing turns.
	  \begin{itemize}
	  \item Cannot maintain same pitch/airspeed as level climb.
	  \item Steep banks take away from climb.
	  \item In a climbing turn, bank should be shallow.
	  \item Should maintain coordinated flight, constant airspeed, constant rate of turn.
	  \end{itemize}
\item Descending turns.
	  \begin{itemize}
	  \item Partial power descent:  typically 500 ft/min.
	  \item Minimum safe airspeed descent:  typically obstacle clearance on landing.
	  \item Emergency descent:  procedure depends on the model of airplane (gear, flap settings, other elements of procedure).
	  \end{itemize}
\item Glides
	  \begin{itemize}
	  \item Use 10 percent of rate of descent as a guide for when to begin leveling.
	  \item Best glide speed:  max L/D.
	  \item Left rudder likely necessary.
	  \item Flight control deflection likely greater, due to reduced airflow over surfaces.
	  \item Minimum sink speed:  a few knots less than the best glide speed, typically.  Seldom published.
	        Appropriate for certain maneuvers like ditching into water.
	  \item Never attempt to stretch a glide.
	  \end{itemize}
\item Gliding turns.
	  \begin{itemize}
	  \item More back pressure on elevator required than during powered flight.
	  \item Less rudder pressure required because of less airflow.  Deflection may be similar, but less
	        pressure required to achieve the deflection.
	  \item Hold nose up until best glide speed is reached, and only then allow further descent.
	  \item Must lower nose slightly at gliding turn entry (must have lesser pitch than glide).
	  \end{itemize}
\end{itemize}


%%%%%%%%%%%%%%%%%%%%%%%%%%%%%%%%%%%%%%%%%%%%%%%%%%%%%%%%%%%%%%%%%%%%%%%%%%%%%%%

\subsection{Notes from Reviewing \emph{Chapter 4:  Maintaining Aircraft Control:
            Upset Prevention and Recovery Training}
            (Reviewed 7/26/2020)}
\label{snra0:supr0}

\begin{itemize}
\item Pilot's fundamental responsibility to to prevent a loss of control.
	  \begin{itemize}
	  \item Definition:  significant deviation of an aircraft from the intended flightpath.
	  \item LOC = loss of control.  LOC-I = loss of control in flight.
	  \item Often results from an airplane upset.
	  \item Leading cause of fatal general aviation accidents in U.S.
	  \item May occur in any phase of flight, but most common during maneuvering.
	  \end{itemize}
\item Upset definition:  event that unintentionally exceeds the parameters normally
      experienced in flight or flight training (pitch attitude outside $[-10, 25]$, bank
	  angle $>45$, flying at airspeeds inappropriate for the conditions).
\item Upset prevention and recovery training (UPRT):  slow flight, stalls, spins, and unusual attitudes.
\item Coordinated flight.
	  \begin{itemize}
	  \item Pilot correcting for yaw effects.
	  \item Pilot should develop a feel for [inappropriate] side loads.
	  \item ``Step on the ball.''
	  \end{itemize}
\item Angle of attack.
	  \begin{itemize}
	  \item Angle at which the chord of the wing meets the relative wind.
	  \item Critical angle (stall):  flow of air separates from the upper surface and backfills, burbles, and eddies,
	        which reduces lift and increases drag.
	  \item Can stall regardless of speed by increasing load on wings (entering a turn, pitching up).
	  \end{itemize}
\item Slow flight.
	  \begin{itemize}
	  \item Airplane is just under the AOA that will cause an aerodynamic buffet or warning
	        from a stall warning device.
	  \item Stall near ground is catastrhophic, so pilot must learn proficiency in recovery.
	  \item Slow flight helps pilot to develop feel and recognition of near-stall characteristics (degraded
	        response to control inputs, difficulty maintaining altitude).
	  \item Degraded response to control inputs---``sloppy'' or ``mushy'' controls.
	  \item Below $L/D_{MAX}$:
    	  \begin{itemize}
	      \item Large power changes required to increase airspeed.
	      \item Small pitch changes result in large drag changes, so pitch becomes more effective control for airspeed,
		        and throttle more effective for climb/descent.
	      \item Speed instability:  perturbation can cause airspeed to continue to decrease.  Decrease leads to decrease,
		        rather than increase.
	      \end{itemize}
	  \end{itemize}
\item Performing the slow flight maneuver.
	  \begin{itemize}
	  \item No lower than 1500 feet AGL, or higher if recommended by the airplane manufacturer.
	  \item Clear the area.
	  \item Reduce power, hold altitude.
	  \item As the speed approaches target speed, add power.
	  \item Trim airplane.
	  \item Typically performed and evaluated in the landing configuration.
	  \item Perform ``before landing'' checks before the maneuver.
	  \item Flaps and gear typically deployed after plane has slowed down from cruise
	        to avoid overspeeding these components.
	  \item May be performed in other configurations besides landing configuration for
	        additional practice.
	  \item Strong right rudder required.
	  \end{itemize}
\item Maneuvering in slow flight.
	  \begin{itemize}
	  \item Necessary to increase power when turning.
      \item Pitch drives airspeed, power drives climb/descent.
	  \end{itemize}
\item Stalls.
	  \begin{itemize}
	  \item Must have adequate altitude (how much?).  Altitude loss is normal during stall recovery
	        practice.
	  \end{itemize}
\item Angle of attack indicators.
	  \begin{itemize}
	  \item Provide visual image of proximity to critical AOA.
	  \end{itemize}
\item Stall characteristics.
	  \begin{itemize}
	  \item Training airplanes: design is usually so that wings stall starting at the wing root,
	        and the stall moves
	        progressively outward.
	  \item Airflow near tips can produce special problems in recovery (down aileron may deepen stall, leading
	        to reverse control action).  Reduce AOA before trying
	        to bank the plane using ailerons.
	  \end{itemize}
\item Fundamentals of stall recovery.
	  \begin{itemize}
	  \item Reduce AOA (pitch), level plane, apply power.
	  \end{itemize}
\item Stall training.
	  \begin{itemize}
	  \item Recovery completed no lower than 1500 feet AGL, or higher if recommended by the airplane
	        manufacturer.
	  \end{itemize}
\item Approaches to stalls (impending stalls), power-on or power-off.
\item Full stalls, power-off.
	  \begin{itemize}
	  \item Normal landing and approach configuration, to simulate stall occurring during normal approach
	        to landing.
	  \item Essentially, recover into go-around configuration, positive rate of climb.
	  \item Should also be practiced from shallow banked turns to simulate an
	        inadvertant stall turning a turn from base leg to final approach.
	  \end{itemize}
\item Full stalls, power-on
\item Secondary stall.
	  \begin{itemize}
	  \item Occurs after recovery from preceding stall.
	  \item Abrupt control inputs, attempting to return to desired flightpath too quickly,
	        not sufficiently reducing AOA, attempting to recover from stall using power only.
	  \item Recovery steps are the same as for primary stall.
	  \end{itemize}
\item Accelerated stall.
	  \begin{itemize}
	  \item Stall at $>$1G, airspeed $>$ normal 1G stalling airspeed, typically during turns or pull-ups from dives.
	        Base to final turn is typical scenario.
	  \end{itemize}
\item Cross-control stall.
	  \begin{itemize}
	  \item Be very reluctant to exceed bank angles of 30 degrees on the base to final turn.  Go-around if there is
	        an overshoot.
	  \end{itemize}
\item Elevator trim stall.
	  \begin{itemize}
	  \item May occur during go-arounds when trim is set for landing approach.
	  \item Increase in power may pitch nose up sharply.
	  \item Must use forward yoke pressure to overcome trim forces, then re-trim eventually.
	  \end{itemize}
\item Spin awareness.
	  \begin{itemize}
	  \item Spins typically occur from a full stall with airplane in a yawed state.
	  \item Airplane may yaw for many reasons, including uncoordinated state, adverse yaw from ailerons,
	        turbulence, etc.
	  \item Rotation results from unequal AOA on the airplane's wings.
	  \item Maintaining directional control and not allowing the nose to yaw
	        before stall recovery is initiated is the key to averting a spin.  The
			pilot must apply the correct amount of rudder to keep the nose from yawing and
			the wings from banking.
	  \end{itemize}
\item Spin procedures.
	  \begin{itemize}
	  \item Not all airplanes are approved for spins, and those that are may be approved
	        only in certain weight/CG parts of the envelope.
	  \item Spin recovery is to some degree airplane-dependent.  Must review POH.
	  \item PARE, except forward elevator is also mentioned, and also neutralizing
	        the rudder.  Need to check on the right procedure.
	  \end{itemize}
\item Intentional spins.
\item Upset prevention and recovery
	  \begin{itemize}
	  \item Unusual attitude.
	  \item Upset--defined parameters, unusual attitude is a subset.
	  \end{itemize}
\item UPRT core concepts
	  \begin{itemize}
	  \item Act early.
	  \item Upset recovery template:
	        \begin{itemize}
			\item Disconnect the wing leveler or autopilot.
       	    \item Apply forward column pressure to unload the airplane.
			\item Aggressive roll the wings to the nearest horizon.
			\item Adjust power as necessary by monitoring airspeed.
			\item Return to level flight.
	        \end{itemize}
	  \item FSTD = ???  Flight simulator training device?
	  \end{itemize}
\end{itemize}


%%%%%%%%%%%%%%%%%%%%%%%%%%%%%%%%%%%%%%%%%%%%%%%%%%%%%%%%%%%%%%%%%%%%%%%%%%%%%%%

\subsection{Notes from Reviewing \emph{Chapter 5:  Takeoffs and Departure
            Climbs}
            (Reviewed 8/1/2020)}
\label{snra0:stdc0}

\begin{itemize}
\item Disproportionate number of accidents occur during takeoff and departure climbs,
      relative to duration of flight.
\item Takeoff roll:
      \begin{itemize}
      \item In tricycle-gear plane, no
            pressures required on elevator beyond those required
            to steady it.
      \item Aileron into crosswind.
	  \end{itemize}
\item Lift-off:
      \begin{itemize}
      \item Assume $V_Y$ immediately after lift-off.
      \item In strong/gusty winds, may want to build more speed before rotating
	        to allow for positive control and positive liftoff.
	  \end{itemize}
\item Initial climb:
      \begin{itemize}
      \item Adjust pitch for $V_Y$ immediately after lift-off.
	  \end{itemize}
\item Crosswind takeoff:
      \begin{itemize}
      \item Initially, full aileron into wind.
	  \item As forward speed increases, apply only enough aileron pressure
	        to keep the airplane laterally aisnged with the runway centerline.
      \item If inadequate aileron, airplane may ``skip''.  This may impose
	        severe stresses on the landing gear and cause damage or structural
			failure.
      \item As nose wheel rises, upwind wheel may be the only one left
	        on the runway.  This is standard technique.
      \item With significant crosswind, hold main wheels on the ground slightly longer
	        for positive lift-off.
      \item Enter slip at lift-off.
      \item Don't fully understand the words at the top-right of 5-8.  Need
	        to confirm phases with CFI.
	  \end{itemize}
\item Short-field takeoff and maximum performance climb.
\item Soft/rough field takeoff and climb.
      \begin{itemize}
      \item Want airplane airborne as quick as possible to eliminate drag from mud, tall
	        grass, etc.
	  \item Techniques make judicious use of ground effect.
      \item Lower nose when wheels clear of surface.
      \item Stay in ground effect and do not try to climb until at least $V_X$.
	  \end{itemize}
\item Rejected takeoff/engine failure:
      \begin{itemize}
      \item Identify a point along the runway at which plane should be airborne.  If not,
	        immediately discontinue the takeoff.
	  \item Reduce power to idle and full braking.
	  \end{itemize}
\end{itemize}

%%%%%%%%%%%%%%%%%%%%%%%%%%%%%%%%%%%%%%%%%%%%%%%%%%%%%%%%%%%%%%%%%%%%%%%%%%%%%%%

\subsection{Notes from Reviewing \emph{Chapter 6:  Ground Reference Maneuvers}
            (Reviewed 8/1/2020)}
\label{snra0:sgrm0}

\begin{itemize}
\item The purpose of ground reference maneuvers is to train pilots to accurately
      place the airplane in relationship to specific references and maintain a desired
	  ground track.
\item Always clear the area with two 90-degree clearing turns looking to the left and right,
      as well as above and below the airplane.
\item Should not exceed a bank angle of 45 degrees or an airspeed greater than maneuvering
      speed.
\item As part of preflight planning, determine the predicted POH/AFM stall speed at 50
      degrees of bank.  (I could do this, but never have.  Should, as an exercise.)
\item Ground reference maneuvers are generally flown at altitudes between 600 and 1,000
      feet above the ground.
\item Due to low altitude, should plan emergency fields.
\item Constant radius during turning flight.
      \begin{itemize}
      \item TBD.
	  \end{itemize}
\end{itemize}

%%%%%%%%%%%%%%%%%%%%%%%%%%%%%%%%%%%%%%%%%%%%%%%%%%%%%%%%%%%%%%%%%%%%%%%%%%%%%%%

\section{Notes from Cessna 172R POH}
\label{snrc0}

TBD.


%%%%%%%%%%%%%%%%%%%%%%%%%%%%%%%%%%%%%%%%%%%%%%%%%%%%%%%%%%%%%%%%%%%%%%%%%%%%%%%

\section{Flight Lessons}
\label{sfle0}

This section contains an possible/proposed outline of things to do and discuss
during flight lessons.  Most of the lessons come from the PPL PTS, but there
are other sources as well.


%%%%%%%%%%%%%%%%%%%%%%%%%%%%%%%%%%%%%%%%%%%%%%%%%%%%%%%%%%%%%%%%%%%%%%%%%%%%%%%

\subsection{Introductory Flight (Completed 7/17/2020)}
\label{sfle0:sint0}


%%%%%%%%%%%%%%%%%%%%%%%%%%%%%%%%%%%%%%%%%%%%%%%%%%%%%%%%%%%%%%%%%%%%%%%%%%%%%%%

\subsubsection{Purpose/Agenda}
\label{sfle0:sint0:spov0}

\begin{itemize}
\item To determine if I'm still comfortable operating a small airplane.
\item To determine how badly my skills have detiorated.
\item To make an informal traning plan with CFI\textsubscript{1}.
\end{itemize}

%%%%%%%%%%%%%%%%%%%%%%%%%%%%%%%%%%%%%%%%%%%%%%%%%%%%%%%%%%%%%%%%%%%%%%%%%%%%%%%

\subsubsection{Notes from Lesson}
\label{sfle0:sint0:snff0}

\begin{itemize}
\item CFI\textsubscript{1} is available on Fridays.
\item A 2-hour appointment is about right.
\item Surprisingly, use of rudder pedals is still acceptable (no difficulty taxiing,
      no weaving on takeoff roll).
\item Difficulty running all 3 control loops during crosswind landing
      (\S{}\ref{sqsa0:slnd0:slhy0}).
\item Difficulty in the last 10 vertical feet, no float 
      (\S{}\ref{sqsa0:slnd0:srdf0}).
\item Airplane is a newer model than I trained on.  Need familiarity with the engine,
      leaning procedures, GPS, and ADS-B.	  
\end{itemize}


%%%%%%%%%%%%%%%%%%%%%%%%%%%%%%%%%%%%%%%%%%%%%%%%%%%%%%%%%%%%%%%%%%%%%%%%%%%%%%%

\subsection{Slow Flight, Stalls (Not Yet Completed)}
\label{sfle0:ssfs0}


%%%%%%%%%%%%%%%%%%%%%%%%%%%%%%%%%%%%%%%%%%%%%%%%%%%%%%%%%%%%%%%%%%%%%%%%%%%%%%%

\subsubsection{Purpose/Agenda}
\label{sfle0:ssfs0:spov0}

\begin{itemize}
\item Slow flight.
\item Power-off stalls and recovery.
\item Power-on stalls and recovery.
\end{itemize}

%%%%%%%%%%%%%%%%%%%%%%%%%%%%%%%%%%%%%%%%%%%%%%%%%%%%%%%%%%%%%%%%%%%%%%%%%%%%%%%

\subsubsection{Notes from Lesson}
\label{sfle0:ssfs0:snff0}

TBD.


%%%%%%%%%%%%%%%%%%%%%%%%%%%%%%%%%%%%%%%%%%%%%%%%%%%%%%%%%%%%%%%%%%%%%%%%%%%%%%%

\subsection{Recovery from Unusual Attitudes (Not Yet Completed)}
\label{sfle0:srua0}


%%%%%%%%%%%%%%%%%%%%%%%%%%%%%%%%%%%%%%%%%%%%%%%%%%%%%%%%%%%%%%%%%%%%%%%%%%%%%%%

\subsubsection{Purpose/Agenda}
\label{sfle0:srua0:spov0}

TBD.

%%%%%%%%%%%%%%%%%%%%%%%%%%%%%%%%%%%%%%%%%%%%%%%%%%%%%%%%%%%%%%%%%%%%%%%%%%%%%%%

\subsubsection{Notes from Lesson}
\label{sfle0:srua0:snff0}

TBD.


%%%%%%%%%%%%%%%%%%%%%%%%%%%%%%%%%%%%%%%%%%%%%%%%%%%%%%%%%%%%%%%%%%%%%%%%%%%%%%%

\subsection{Engine Failures (Not Yet Completed)}
\label{sfle0:sefa0}


%%%%%%%%%%%%%%%%%%%%%%%%%%%%%%%%%%%%%%%%%%%%%%%%%%%%%%%%%%%%%%%%%%%%%%%%%%%%%%%

\subsubsection{Purpose/Agenda}
\label{sfle0:sefa0:spov0}

\begin{itemize}
\item Engine failures pre-$V_1$/$V_2$.
\item Engine failures post-$V_1$/$V_2$ and below 800 feet.
\item Engine failures above 800 feet and below cruise.
\item Engine failures in cruise.
\item Engine failures in airport approach.
\item Engine failures in pattern.
\end{itemize}

%%%%%%%%%%%%%%%%%%%%%%%%%%%%%%%%%%%%%%%%%%%%%%%%%%%%%%%%%%%%%%%%%%%%%%%%%%%%%%%

\subsubsection{Notes from Lesson}
\label{sfle0:sefa0:snff0}

TBD.

%%%%%%%%%%%%%%%%%%%%%%%%%%%%%%%%%%%%%%%%%%%%%%%%%%%%%%%%%%%%%%%%%%%%%%%%%%%%%%%

\subsection{Control Failures (Not Yet Completed)}
\label{sfle0:scfa0}


%%%%%%%%%%%%%%%%%%%%%%%%%%%%%%%%%%%%%%%%%%%%%%%%%%%%%%%%%%%%%%%%%%%%%%%%%%%%%%%

\subsubsection{Purpose/Agenda}
\label{sfle0:scfa0:spov0}

\begin{itemize}
\item Throttle cable/linkage failure.
\item Mixture cable/linkage failure.
\item Brake failure.
\item Elevator failure.
\item Trim tab failure.
\item Rudder failure.
\item Aileron failure.
\end{itemize}

%%%%%%%%%%%%%%%%%%%%%%%%%%%%%%%%%%%%%%%%%%%%%%%%%%%%%%%%%%%%%%%%%%%%%%%%%%%%%%%

\subsubsection{Notes from Lesson}
\label{sfle0:scfa0:snff0}

TBD.

%%%%%%%%%%%%%%%%%%%%%%%%%%%%%%%%%%%%%%%%%%%%%%%%%%%%%%%%%%%%%%%%%%%%%%%%%%%%%%%

\section{Notes about Obtaining Third-Class Medical Certificate}
\label{stcm0}

TBD.

%%%%%%%%%%%%%%%%%%%%%%%%%%%%%%%%%%%%%%%%%%%%%%%%%%%%%%%%%%%%%%%%%%%%%%%%%%%%%%%

\section{Questions (and Some Answers)}
\label{sqsa0}

%CFI\textsubscript{1}

%%%%%%%%%%%%%%%%%%%%%%%%%%%%%%%%%%%%%%%%%%%%%%%%%%%%%%%%%%%%%%%%%%%%%%%%%%%%%%%

\subsection{Flying in General, Cruise Flight}
\label{sqsa0:sfig0}

%%%%%%%%%%%%%%%%%%%%%%%%%%%%%%%%%%%%%%%%%%%%%%%%%%%%%%%%%%%%%%%%%%%%%%%%%%%%%%%

\subsubsection{Why Was CFI\textsubscript{1} Less Than Happy With ``Your Airplane''?}
\label{sqsa0:sfig0:slhy0}

\emph{Question:} Noticed on my initial flight that there seemed to be some
displeasure with ``your airplane''?  What was the source of displeasure, and what
is the best procedure for positive exchange of flight controls?

\noindent{}\emph{Answer:}

%%%%%%%%%%%%%%%%%%%%%%%%%%%%%%%%%%%%%%%%%%%%%%%%%%%%%%%%%%%%%%%%%%%%%%%%%%%%%%%

\subsubsection{Where is ADS-B Required?}
\label{sqsa0:sfig0:sads0}

\index{ADS-B}\emph{Question:} During initial flight, the instructor indicated that the NAV
light switch also controlled ADS-B.  On my sectional chart, noticed the 30NM
radius where a MODE-C transponder is required, but didn't see a mention of ADS-B.
Where/when is it required?

\noindent{}\emph{Answer:}


%%%%%%%%%%%%%%%%%%%%%%%%%%%%%%%%%%%%%%%%%%%%%%%%%%%%%%%%%%%%%%%%%%%%%%%%%%%%%%%

\subsubsection{Which Spin Recovery Procedure Should I Memorize?}
\label{sqsa0:sfig0:swsr0}

When I trained, the procedure was \emph{PARE}, power off, ailerons neutral,
rudder smartly against the spin, elevator forward.  The AFH has a more elaborate
procedure with two extra steps.  Which should I memorize?

\noindent{}\emph{Answer:}


%%%%%%%%%%%%%%%%%%%%%%%%%%%%%%%%%%%%%%%%%%%%%%%%%%%%%%%%%%%%%%%%%%%%%%%%%%%%%%%

\subsubsection{What is the Best Procedure / Method for Clearning Turns?}
\label{sqsa0:sfig0:sclt0}

\emph{Question:} The AFH makes many references to clearing procedures.  Suggestions?

\noindent{}\emph{Answer:}

%%%%%%%%%%%%%%%%%%%%%%%%%%%%%%%%%%%%%%%%%%%%%%%%%%%%%%%%%%%%%%%%%%%%%%%%%%%%%%%

\subsection{Landing}
\label{sqsa0:slnd0}

%%%%%%%%%%%%%%%%%%%%%%%%%%%%%%%%%%%%%%%%%%%%%%%%%%%%%%%%%%%%%%%%%%%%%%%%%%%%%%%

\subsubsection{Exercises for Missing Control Loops on Initial Flight}
\label{sqsa0:slnd0:slhy0}

\emph{Question:} Noticed on my initial flight that I was so preoccupied with
elevator control (let's call this \emph{Control Loop No. 1}, or \emph{CL1})
that I neglected CL2 (bank angle) and CL3 (yaw).  I'm not especially surprised,
because I hadn't flown in 15 years, and because of what is described in
\S{}\ref{sqsa0:slnd0:srdf0}.  Are there any exercises for getting the control loops
back running again.

\noindent{}\emph{Answer from CFI\textsubscript{3}:}
CFI\textsubscript{3} described exercises s/he had done with students where she
would have them fly down a runway on a day with a crosswind at low speed and
have the student control the bank angle (with s/he
handling the rudders and watching for unsafe conditions), then on another
pass with the student handling the rudders.  S/he had a method she had used
to reinforce CL2 and CL3 separately.

My idea for this is to land with less than full flaps and partial power to give
myself more time (very similar to the answer from CFI\textsubscript{3}).


%%%%%%%%%%%%%%%%%%%%%%%%%%%%%%%%%%%%%%%%%%%%%%%%%%%%%%%%%%%%%%%%%%%%%%%%%%%%%%%

\subsubsection{Reason for Rapid Descent from 10 Feet to Runway (No Float)?}
\label{sqsa0:slnd0:srdf0}

\emph{Question:} Noticed on my initial flight that plane sank from 10 feet
to runway quite quickly.  There was no float.  This was a contributor to
what is described in \S{}\ref{sqsa0:slnd0:slhy0}.  Why was there no float?

\noindent{}\emph{Answer from CFI\textsubscript{3}:}  Suggested that:
\begin{itemize}
\item Airspeed may have been too low.
\item May have flared too high.
\end{itemize}

%%%%%%%%%%%%%%%%%%%%%%%%%%%%%%%%%%%%%%%%%%%%%%%%%%%%%%%%%%%%%%%%%%%%%%%%%%%%%%%

\subsection{Cessna 172R and N3544B}
\label{sqsa0:sc7r0}

%%%%%%%%%%%%%%%%%%%%%%%%%%%%%%%%%%%%%%%%%%%%%%%%%%%%%%%%%%%%%%%%%%%%%%%%%%%%%%%

\subsubsection{Why Was There no Vacuum Check During Run-Up?}
\label{sqsa0:sc7r0:slhy0}

\emph{Question:} This airplane has conventional flight instruments (driven by
vacuum and electricity).  Why was there no check of vacuum during the runup?  Is
there a vacuum gauge?

\noindent{}\emph{Answer:}


%%%%%%%%%%%%%%%%%%%%%%%%%%%%%%%%%%%%%%%%%%%%%%%%%%%%%%%%%%%%%%%%%%%%%%%%%%%%%%%
%Bibliography
%\cleardoublepage
%\addcontentsline{toc}{chapter}{Bibliography}
%\input{misc_latex/workbibl}
%
%%%%%%%%%%%%%%%%%%%%%%%%%%%%%%%%%%%%%%%%%%%%%%%%%%%%%%%%%%%%%%%%%%%%%%%%%%%%%%%
%Index Must Be Formed At This Directory Level
\clearpage
\addcontentsline{toc}{section}{Index}
\printindex

\end{document}
